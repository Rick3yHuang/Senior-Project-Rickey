\documentclass{article}

% Language setting
% Replace `english' with e.g. `spanish' to change the document language
\usepackage[english]{babel}

% Set page size and margins
% Replace `letterpaper' with`a4paper' for UK/EU standard size
\usepackage[letterpaper,top=2cm,bottom=2cm,left=3cm,right=3cm,marginparwidth=1.75cm]{geometry}

% Useful packageshttps://www.overleaf.com/project/617db8234c85bd8404625de7
\usepackage{amsmath}
\usepackage{graphicx}
\usepackage{amsthm}
\usepackage{amssymb}
\usepackage[colorlinks=true, allcolors=blue]{hyperref}
\usepackage[dvipsnames]{xcolor}
\newtheorem{theorem}{Theorem}[section]
\newtheorem{corollary}{Corollary}[theorem]
\newtheorem{lemma}[theorem]{Lemma}
\usepackage{float}
\usepackage[dvipsnames]{xcolor}
\usepackage{algorithm}
\usepackage{algpseudocode}

\usepackage[
backend=biber,
style=alphabetic,
sorting=ynt
]{biblatex}

\usepackage{alltt}

\addbibresource{research.bib}

\theoremstyle{definition}
\newtheorem{definition}{Definition}[section]

\theoremstyle{remark}
\newtheorem*{remark}{\textbf{Remark}}

\theoremstyle{example}
\newtheorem{example}{\textbf{Example}}[section]

\title{Computing Minimal Free Schreyer Resolution\\
\begin{large} 
  Senior Project
\end{large}}
\author{Ruiqi (Rickey) Huang}

\newcommand{\qedwhite}{\hfill \ensuremath{\Box}}

\begin{document}
\maketitle

\begin{abstract}
    \textcolor{BrickRed}{waiting for editing}
\end{abstract}

\section{Introduction}

\paragraph{  }

\textcolor{BrickRed}{waiting for editing}

\paragraph{  }

\section{Preliminaries}

\paragraph{  }

We define the least common multiple monomials in the following way
\begin{definition}[\textbf{Least Common Multiple}]
    Let $f, g \in k[x_1, \cdots, x_n]$ be nonzero polynomials such that $multideg(f) = \alpha$ and $multideg(g) = \beta$, then let $\gamma = (\gamma_1, \cdots, \gamma_n)$, where $\gamma_i = max(\alpha_i,\beta_i)$. Then $x^{\gamma}$ is the least common multiple of $LM(f)$ and $LM(g)$. That is $x^{\gamma} = lcm(LM(f), LM(g))$.
\end{definition}

\paragraph{  }
To cancel the leading terms of two polynomials, we need the help from the definition below.
\begin{definition}[\textbf{S-polynomial}]
    Let $f, g \in k[x_1, \cdots, x_n]$ be nonzero polynomials, and let $x^{\gamma} = lcm(LM(f), LM(g))$ then the S-polynomial of $f$ and $g$ is defined as a polynomial that cancels leading terms of both the polynomials. to be specific.
    \begin{equation}
        S(f,g) = \tfrac{x^{\gamma}}{LT(f)} \cdot f - \tfrac{x^{\gamma}}{LT(g)}\cdot g
    \end{equation}
\end{definition}

\paragraph{  }
We can easy see the cancellation of the leading terms simply via the following calculation:

Let $f' = f - LT(f)$ and $g' = g - LT(g)$. Then:
\begin{align}
    \tfrac{x^{\gamma}}{LT(f)} \cdot f - \tfrac{x^{\gamma}}{LT(g)} \cdot g &= \tfrac{x^{\gamma}}{(LT(f) + f')} - \tfrac{x^{\gamma}}{(LT(g) + g')}\\
    &= x^{\gamma} + \tfrac{x^{\gamma}}{LT(f)} \cdot f' - x^{\gamma} - \tfrac{x^{\gamma}}{LT(g)} \cdot g'\\
    &= \tfrac{x^{\gamma}}{LT(f)} \cdot f' - \tfrac{x^{\gamma}}{LT(g)} \cdot g' 
\end{align}

Clearly, the S-polynomial's degree drops by canceling the leading terms.

\begin{lemma}
    For a sum $\sum_{i = 1}^{s}{p_i}$, where $multideg(p_i) = \delta \in Z_{\geq0}^{n}$ for all $i$, if $multideg(\sum_{i = i}^{s}{p_i}) < \delta$, then $\sum_{i = 1}^{s}{p_i}$ is a linear combination, with coefficients in $k$, of the S-polynomials $S(p_j, p_l)$ for $1 \leq j,l \leq s$. Furthermore, each $S(p_j, p_l)$ has multidegree $< \delta$.
\end{lemma}

\textcolor{BrickRed}{include the proof later}

\begin{theorem}[\textbf{Buchberger's Criterion}]
    Let $I = <f_1,\cdots,f_s>\neq \{0\}$ be a polynomial ideal. Then a Gröbner basis for I can be constructed in a finite number of steps by the following algorithm:
    \begin{algorithm}[H]
        \caption{Buchberger's Criterion}\label{alg:bc}
        \begin{alltt}
        Input: F = \((f\sb{1},\cdots,f\sb{s})\)
        Output: a Gröbner basis \(G=(g\sb{1},\cdots,g\sb{t})\) for \(I\), with \(F \subseteq G\)
        
        \(G:=F\)
        REPEAT
        \quad \quad \quad \(G':=G\)
        \quad \quad \quad FOR each pair\(\{p,q\}, p \neq q\) in \(G'\) DO
        \quad \quad \quad \quad \quad \quad \(r := {\overline{S(p,q)}}\sp{G'}\)
        \quad \quad \quad \quad \quad \quad IF \(r\neq 0\) THEN \(G\:=G\cup\{r\}\)
        UNTIL \(G=G'\)
        RETURN \(G\)
    \end{alltt}
    \end{algorithm}
\end{theorem}

\begin{proof}
\begin{enumerate}
    \item Let $I$ be a polynomial ideal. Then a basis $G = \{g_1, \cdots, g_t\}$ of $I$ is a Gröbner basis of $I$ if and only if for all pairs $i \neq j$, the remainder on division of $S(g_i, g_j)$ by $G$ (listed in some order) is zero.
    \item This algorithm terminates because of the ACC ($LT(G') \subseteq LT(G)$)
\end{enumerate}
\end{proof}

\begin{remark}
    The Buchberger's algorithm only produces a Gröbner Basis which is not unique, and it might include redundancy.
\end{remark}

\begin{definition}[\textbf{Minimal Gröbner Basis}]
    The minimal Gröbner basis is a Gröbner basis  $G$ for $I$ such that:
    \begin{itemize}
        \item $LC(p) = 1$ for all $p \in G$
        \item For all $p \in G$, none of $LT(p)$ lies in $<LT(G\backslash\{p\})>$
    \end{itemize}
\end{definition}

\begin{definition}[\textbf{Reduced Gröbner Basis}]
    The reduced Gröbner basis is a Gröbner basis  $G$ for $I$ such that:
    \begin{itemize}
        \item $LC(p) = 1$ for all $p \in G$
        \item For all $p \in G$, none of $LT(p)$ lies in $<LT(G\backslash\{p\})>$
    \end{itemize}
\end{definition}

\begin{remark}
    The row reduction is a special case for buchberger's algorithm
\end{remark}

\subsection{Gröbner Bases}

\begin{definition}[\textbf{Gröbner Basis}]\cite{cox_grobner_2015}\label{def:gb}
    Fix a monomial order on the polynomial ring $k[x_1, \cdots, x_n]$. A finite subset $G = \{g_1, \cdots, g_t\}$ of an ideal $I \subseteq k[x_1, \cdots, x_n]$ different from $\{0\}$ is said to be \textbf{Gröbner basis} (or \textbf{standard basis}) if
    \begin{center}
        $<LT(g_1), \cdots, LT(g_t)> = <LT(I)>$
    \end{center}
\end{definition}



\begin{theorem}[\textbf{Hilbert Basis Theorem}]\cite{cox_grobner_2015}\label{thm:hb_thm}
    Every ideal $I \subseteq k[x_1,\cdots,x_n]$ has a finite generating set. In the other words, $I = <g_1,\cdots,g_t>$, for some $g_1,\cdots,g_t \in I$.
\end{theorem}

Theorem \ref{thm:hb_thm} implies that every ideal $I \subseteq k[x_1, \cdots, x_n]$ has a finite generating set. In other words, $I = <g_1, \cdots, g_t>$ for some $g_1, \cdots, g_t \in I$. Thus, we have the following corollary.

\begin{corollary}\cite{cox_grobner_2015}\label{cor:gbI}
    Fix a monomial order on the polynomial ring $k[x_1,\cdots,x_n]$,every ideal in $k[x_1, \cdots, x_n]$ has a Gröbner basis. Additionally, any Gröbner basis for an ideal $I$ is a basis of $I$.
\end{corollary}

\subsection{Properties of Gröbner Bases}
\begin{itemize}
    \item $<LT(I)>$ is strictly larger than $<LT(g_1), \cdots, LT(g_s)>$
    \item Dividing $f$ by a Gröbner basis ($G = \{g_1, g_2, \cdots, g_t\}$), the remainder would be unique, but the quotient $q$ would be different if the generators of $G$ are list in a different order.
    \item $f \in I$ if and only if $0$ is a remainder of $f$ on division by $G$
    \item $\overline{f}^{F}$ is the remainder on division of $f$ by the ordered s-tuple $F = (f_1, f_2, \cdots, f_s)$
\end{itemize}

\begin{remark}
    An obstruction to a basis be a Gröbner basis is that the cancellation occur in $f_k = ax^{\alpha}f_i - bx^{\beta}f_{j}$, where $f_i,f_j \in I$, and $f_k \in <LT(I)>$, but $f_k \notin <LT(f_i), LT(f_j)>$.
\end{remark}

- a power product (or monomial of S): $t = x_1^{\alpha_1}\cdots x_n^{\alpha_n} \in S$, where $\alpha_1, \cdots, \alpha_n$ are non-negative integers
- for any set $G \subseteq S$, we let $in(G)$ denote the $K$-vector space spanned by the monomials $\{lm(f):f \in G\}$.
- If $J \subset S$ is an ideal, then $in(J) \subset S$ is the monomial ideal generated by the lead terms of $J$
- Let $R = S/J$ be a factor ring of $S$. Let $N$ denotes the $K$-vector space spanned by the set of standard monomials of $R$, that is, the set of monomials of $S$ not in $in(J)$. Any element $f \in R$ may be uniquely written as the image of an element $g \in N$, and we set $lc(f) = lc(g)$ and $lm(f) = lm(g) \in N$.
- For any set $G \subset R$, let $in(G)$ denote the $K$-vectors space generated by the lead monomials $\{lm(f):f\in G\}$. Thus, $N = in(R)$, and $in(S) = N \bigoplus in(J)$, as $K$-vector spaces.
- Let $F$ be a free module over $R$, and let $\hat{F}$ be the free $S$-module with the same rank as $F$ and corresponding basis. A **monomial** of $F$ is by definition any element $m = t \cdot e$, where $t \in N$ is a standard monomial and $e$ is any element of the canonical basis of $\hat{F}$.
- **Term order on** $F$ (a total order on the monomials of $F$) such that:
    - if $m <_F n$, then $t\cdot m <_F t \cdot n$;
    - if $s <_S t$, then $s\cdot e <_F t \cdot e$;
- for all $m, n$ monomials of $F$, $s,t$ power products in $S$, and $e$ any basis element of $F$.
- $\{g_1, \cdots, g_s\} \subset I\subset F$ is a Gröbner basis of the $R$-module I, if $\{lm(g_1), \cdots, lm(g_s)\}$ generates $in(I)$, that is, every monomial in $in(I)$ is divisible by some $lm(g_i)$.
- auto-reduced Gröbner basis: every lead monomial $lm(g_i)$ divides no monomial occuring in ang $g_j$ other than itself
- irredundant Gröbner basis: $in(I)$ is minimally generated by $\{lm(g_1), \cdots, lm(g_s)\}$, which in turn means that this set generates $in(I)$, and no lead term $lm(g_i)$ divides any $lm(g_j)$ for $j \neq i$

\section{Schreyer Frame and Schreyer Resolution}

\begin{algorithm}[H]
\caption{Resolution}\label{alg:res}
    \begin{alltt}
        Input: an irredundant Gröbner basis \(\bar{{\mathcal{C}}\sb{1}}\) of \(I\), 
        and an ordered union \(\mathcal{B}\) of bases of all levels
        Output: the set of Gröbner bases \(\mathcal{C}\sb{1},\cdots,\mathcal{C}\sb{l}\),
        and the set of corresponding syzygies \(\mathcal{H}\sb{1},\cdots,\mathcal{H}\sb{l}\)
        
        Resolution\([\bar{{\mathcal{C}}\sb{1}}]\)
        \({\mathcal{C}}\sb{i},{\mathcal{H}}\sb{i}:=\emptyset (1\leq i\leq l)\)
        while \(\mathcal{B}\neq\emptyset\) do
        \quad \quad \(m:=min\mathcal{B}\)
        \quad \quad \(\mathcal{B}:=\mathcal{B}\backslash\{m\}\)
        \quad \quad \(i:=lev(m)\)
        \quad \quad if \(i=1\) then
        \quad \quad \quad \quad \(g:=\) the element of \(\bar{{\mathcal{C}}\sb{i}}\) s.t. \(lm(g)=m\)
        \quad \quad \quad \quad \({\mathcal{C}}\sb{1}:={\mathcal{C}}\sb{1}\cup\{g\}\)
        \quad \quad \quad \quad \({\mathcal{H}}\sb{1}:={\mathcal{H}}\sb{i}\cup\{g\}\)
        \quad \quad else
        \quad \quad \quad \quad \((f,g):=\)Reduce\([m,{\mathcal{C}}\sb{i-1}]\)
        \quad \quad \quad \quad \({\mathcal{C}}\sb{i}:={\mathcal{C}}\sb{i}\cup\{g\}\)
        \quad \quad \quad \quad if \(f\neq 0\) then
        \quad \quad \quad \quad \quad \quad \({\mathcal{C}}\sb{i-1}:={\mathcal{C}}\sb{i-1}\cup\{f\}\)
        \quad \quad \quad \quad \quad \quad \(\mathcal{B}:=\mathcal{B}\backslash\{lm(f)\}\)
        \quad \quad \quad \quad else
        \quad \quad \quad \quad \quad \quad\({\mathcal{H}}\sb{i}:={\mathcal{H}}\sb{i}\cup\{g\}\)
        return \({\mathcal{C}}\sb{i},{\mathcal{H}}\sb{i}(1\leq i\leq l)\)
\end{alltt}
\end{algorithm}

The Algorithm \verb+Reduce+ in \verb+Resolution+ is elaborated in the Algorithm \ref{alg:red}.

\begin{algorithm}
\caption{Reduce[$t\cdot \epsilon,\mathcal{C}_{i-1}]$}\label{alg:red}
    \begin{algorithmic}
    \State $f:=t\cdot k$, where $\varphi_{i-1}(\epsilon) = k$
    \State $g:=t\cdot \epsilon$
    \While{$f \neq 0$ and $lm(f) \in in\langle\mathcal{C}_{i-1}\rangle$}
        \State choose  $h \in \mathcal{C}_{i-1}$ s.t. $lm(h)\vert lm(f)$  \textbf{($h \neq f$ at the first itration)}
        \State $f := f - \tfrac{lc(f)lpp(f)}{lc(h)lpp(h)}h$
        \State $g := g - \tfrac{lc(f)lpp(f)}{lc(h)lpp(h)}e$, where $\varphi_{i-1}(e) = h$
    \EndWhile
    \If{$f \neq 0$}
        \State $g := g - e$, where $\varphi_{i-1}(e) = f$
    \EndIf
    \State return $f,g$
    \end{algorithmic}
\end{algorithm}

\begin{lemma}\label{lem:mingens}
    $in(ker\xi_i)$ is minimally generated by
    \begin{equation}
        \bigcup_{e\in{\varepsilon}_{i-1}}{\bigcup_{j = 2}^{r}{mingens((in(J),t1,\cdots,t_{j-1}):t_j)\cdot \epsilon_j}}
    \end{equation}
    where $\varepsilon_i(e) = \{\epsilon_1,\cdots,\epsilon_r\}$
    \textcolor{BrickRed}{edit more}
\end{lemma}

\begin{example}
    Let $K$ be a field of any characteristic and let the polynomial ring $S = K[x_1,\cdots,x_6]$ with the term order of \verb+DegRevLex+. Consider the graded module $M = S/I$, for 
    \begin{equation}
        I = <x_2x_3,x_2x_4,x_5x_6,x_1x_2,x_4x_6,x_2x_5,x_1x_3x_5,x_1x_3x_6,x_3x_4x_5>
    \end{equation} 
    Using the algorithm \verb+Rsolution+ for $char(K) \neq 2$ to the module $M$, find the minimal Schreyer resolution.
\end{example}

\paragraph{  }

\textbf{Solution:} We first want to compute the Schreyer frame $\Xi$ for $M$. Starting from this angle, the basis of the first level of the Schreyer frame is $I$, because it generates the kernel of the map $F_0 \rightarrow M$. Hence, we have 
\begin{equation}
    \mathcal{B}_1 = \{x_2x_3,x_2x_4,x_5x_6,x_1x_2,x_4x_6,x_2x_5,x_1x_3x_5,x_1x_3x_6,x_3x_4x_5\}
\end{equation}
After ordering it using the \verb+DegRevLex+, the first level basis is:
\begin{equation}
    \mathcal{B}_1 = \{x_5x_6,x_4x_6,x_2x_5,x_1x_4,x_2x_3,x_1x_2,x_1x_3x_6,x_3x_4x_5,x_1x_3x_5\}
\end{equation}
Then, we used the formula in the Lemma \ref{lem:mingens}, to construct the basis of the next level by finding the monomials that knock a basis into the ideal once at a time.
\begin{align}
    j &= 2: (x_5x_6:x_4x_6) = \{x_5\} \rightarrow \{x_5e_{1,2}\}\\
    j &= 3: ((e_{1,1},e_{1,2}):x_2x_5) = \{x_6,x_4x_6\} = \{x_6\} \rightarrow \{x_6e_{1,3}\}\\
    j &= 4: ((e_{1,1},\cdots,e_{1,3}):x_1x_4) = \{x_5x_6,x_6,x_2x_5\} = \{x_6,x_2x_5\} \rightarrow \{x_6e_{1,4},x_2x_5e_{1,4}\}\\
    j &= 5: ((e_{1,1},\cdots,e_{1,4}):x_2x_3) = \{x_5x_6,x_4x_6,x_5,x_1x_4\} = \{x_5,x_4x_6,x_1x_4\}\\
    &\rightarrow \{x_5e_{1,5},x_4x_6e_{1,5},x_1x_4e_{1,5}\}\\
    j &= 6: ((e_{1,1},\cdots,e_{1,5}):x_1x_2) = \{x_5x_6,x_4x_6,x_5,x_4,x_3\} = \{x_5,x_4,x_3\}\\
    &\rightarrow \{x_5e_{1,6},x_4e_{1,6},x_3e_{1,6}\}\\
    j &= 7: ((e_{1,1},\cdots,e_{1,6}):x_1x_3x_6) = \{x_5,x_4,x_2x_5,x_4,x_2,x_2\} = \{x_5,x_4,x_2\}\\
    &\rightarrow \{x_5e_{1,7},x_4e_{1,7},x_2e_{1,7}\}\\
    j &= 8: ((e_{1,1},\cdots,e_{1,7}):x_3x_4x_5) = \{x_6,x_6,x_2,x_1,x_2,x_1x_2,x_1x_6\} = \{x_6,x_2,x_1\}\\
    &\rightarrow \{x_6e_{1,8},x_2e_{1,8},x_1e_{1,8}\}\\
    j &= 9: ((e_{1,1},\cdots,e_{1,8}):x_1x_3x_5) = \{x_6,x_4x_6,x_2,x_4,x_2,x_2,x_6,x_4\} = \{x_6,x_4,x_2\}\\
    &\rightarrow \{x_6e_{1,9},x_4e_{1,9},x_2e_{1,9}\}
\end{align}
Therefore ,we have our second level basis of $\Xi$:
\begin{equation}
    \begin{aligned}
        \mathcal{B}_2 = \{&x_5e_{1,2},x_6e_{1,3},x_6e_{1,4},x_2x_5e_{1,4},x_5e_{1,5},x_4x_6e_{1,5},x_1x_4e_{1,5},x_5e_{1,6},x_4e_{1,6},x_3e_{1,6},x_5e_{1,7},\\
        & x_4e_{1,7},x_2e_{1,7},x_6e_{1,8},x_2e_{1,8},x_1e_{1,8},x_6e_{1,9},x_4e_{1,9},x_2e_{1,9}\}
    \end{aligned}
\end{equation}
After ordering it using the \verb+DegRevLex+, the second level basis is:
\begin{equation}
    \begin{aligned}
        \mathcal{B}_2 = \{&x_5e_{1,2},x_6e_{1,3},x_6e_{1,4},x_5e_{1,5},x_5e_{1,6},x_4e_{1,6},x_3e_{1,6},x_6e_{1,8},x_5e_{1,7},x_6e_{1,9},x_4x_6e_{1,5},x_4e_{1,7},\\
        & x_2e_{1,7},x_2e_{1,8},x_1e_{1,8},x_4e_{1,9},x_2x_5e_{1,4},x_2e_{1,9},x_1x_4e_{1,5}\}
    \end{aligned}
\end{equation}
Renaming each basis with second level subscripts and applying the Lemma \ref{lem:mingens}, we have
\begin{align}
    e_{1,4}: j &= 2: (x_6,x_2x_5) = \{x_6\} \rightarrow\{x_6e_{2,17}\}\\
    e_{1,5}: j &= 2: (x_5,x_4x_6) = \{x_5\} \rightarrow\{x_5e_{2,11}\}\\
    e_{1,5}: j &= 3: ((x_5,x_4x_6):x_1x_4) = \{x_5,x_6\} \rightarrow\{x_6e_{2,19},x_5e_{2,19}\}\\
    e_{1,6}: j &= 2: (x_5:x_4) = \{x_5\} \rightarrow\{x_5e_{2,6}\}\\
    e_{1,6}: j &= 3: ((x_5,x_4):x_3) = \{x_5,x_4\} \rightarrow\{x_5e_{2,7},x_4e_{2,7}\}\\    
    e_{1,7}: j &= 2: (x_5,x_4) = \{x_5\} \rightarrow\{x_5e_{2,12}\}\\
    e_{1,7}: j &= 3: ((x_5,x_4):x_2) = \{x_5,x_4\} \rightarrow\{x_5e_{2,13},x_4e_{2,13}\}\\
    e_{1,8}: j &= 2: (x_6,x_2) = \{x_6\} \rightarrow\{x_6e_{2,14}\}\\
    e_{1,8}: j &= 3: ((x_6,x_2):x_1) = \{x_6,x_2\} \rightarrow\{x_6e_{2,15},x_2e_{2,15}\}\\
    e_{1,9}: j &= 2: (x_6,x_4) = \{x_6\} \rightarrow\{x_6e_{2,16}\}\\
    e_{1,9}: j &= 3: ((x_6,x_4):x_2) = \{x_6,x_4\} \rightarrow\{x_6e_{2,18},x_4e_{2,18}\}
\end{align}

Then, we have the third level basis $\mathcal{B}_3$ of $\Xi$:

\begin{equation}
    \begin{aligned}
        \mathcal{B}_3 = \{&x_6e_{2,17},x_5e_{2,11},x_6e_{2,19},x_5e_{2,19},x_5e_{2,6},x_5e_{2,7},x_4e_{2,7},x_5e_{2,12},x_5e_{2,13},x_4e_{2,13},x_6e_{2,14},\\
        &x_6e_{2,15},x_2e_{2,15},x_6e_{2,16},x_6e_{2,18},x_4e_{2,18}\}
    \end{aligned}
\end{equation}
After ordering it using the \verb+DegRevLex+, the third level basis is:
\begin{equation}
    \begin{aligned}
        \mathcal{B}_3 = \{&x_5e_{2,6},x_5e_{2,7},x_4e_{2,7},x_5e_{2,11},x_6e_{2,14},x_5e_{2,12},x_6e_{2,15},x_6e_{2,16},x_6e_{2,17},x_5e_{2,13},x_6e_{2,18},\\
        & x_6e_{2,19},x_4e_{2,13},x_5e_{2,19},x_2e_{2,15},x_4e_{2,18}\}
    \end{aligned}
\end{equation}

Applying the Lemma \ref{lem:mingens} once again, we find the following bases:
\begin{align}
    e_{2,19}: j &= 2: (x_6,x_5) = \{x_6\} \rightarrow\{x_6e_{3,14}\}\\
    e_{2,7}: j &= 2: (x_5,x_4) = \{x_5\} \rightarrow\{x_5e_{3,3}\}\\
    e_{2,13}: j &= 2: (x_5,x_4) = \{x_5\} \rightarrow\{x_5e_{3,13}\}\\
    e_{2,15}: j &= 2: (x_6,x_2) = \{x_6\} \rightarrow\{x_6e_{3,15}\}\\
    e_{2,18}: j &= 2: (x_6,x_4) = \{x_6\} \rightarrow\{x_6e_{3,16}\}\\
\end{align}

The fourth level basis $\mathcal{B}_4$ of $\Xi$ is

\begin{equation}
    \begin{aligned}
        \mathcal{B}_4 = &\{x_6e_{3,14},x_5e_{3,3},x_5e_{3,13},x_6e_{3,15},x_5e_{3,16}\}
    \end{aligned}
\end{equation}
After ordering it using the \verb+DegRevLex+, the last level basis is:
\begin{equation}
    \begin{aligned}
        \mathcal{B}_4 = &\{x_5e_{3,3},x_6e_{3,14},x_5e_{3,13},x_6e_{3,15},x_5e_{3,16}\}
    \end{aligned}
\end{equation}

Now, applying the Lemma \ref{lem:mingens} could not give us any basis from $\mathcal{B}_4$, then we know that we had found the complete Schreyer frame $\Xi$.

Given the frame $\Xi$, we could implement the Algorithm \ref{alg:res} to construct a Schreyer resolution $\Phi$ corresponding to the frame. To start the algorithm, we should first identify the input by checking whether the first level of our frame gives a Gröbner basis. \textcolor{BrickRed}{Add more to explain the set of bases in the first level $\bar{\mathcal{C}_1}$ is a Gröbner basis}. Then we want to order the union of all bases we obtained in the frame $\Xi$.

\begin{equation}
    \begin{aligned}
        \mathcal{B} = \{&x_5x_6,x_4x_6,x_2x_5,x_1x_4,x_2x_3,x_1x_2,x_5e_{1,2},x_6e_{1,3},x_6e_{1,4},x_1x_3x_6,x_3x_4x_5,x_5e_{1,5},x_1x_3x_5, \\
                    & x_5e_{1,6},x_4e_{1,6},x_3e_{1,6},x_6e_{1,8},x_5e_{1,7},x_6e_{1,9},x_4x_6e_{1,5},x_4e_{1,7},x_2e_{1,7},x_2e_{1,8},x_1e_{1,8},x_4e_{1,9},\\
                    & x_2x_5e_{1,4},x_5e_{2,6},x_2e_{1,9},x_5e_{2,7},x_1x_4e_{1,5},x_4e_{2,7},x_5e_{2,11},x_6e_{2,14},x_5e_{2,12},x_6e_{2,15},x_6e_{2,16},\\
                    & x_6e_{2,17},x_5e_{2,13},x_6e_{2,18}, x_6e_{2,19},x_4e_{2,13},x_5e_{2,19},x_2e_{2,15},x_4e_{2,18},x_5e_{3,3},x_6e_{3,14},x_5e_{3,13},\\
                    & x_6e_{3,15},x_5e_{3,16}\}
    \end{aligned}
\end{equation}

Given $\bar{\mathcal{C}_1}$ and $\mathcal{B}$, we could apply the algorithm \verb+Resolution+. Since in the Algorithm \ref{alg:res}, we just add the first-level bases in $\mathcal{B}$ to $\mathcal{C}_1$ and $\mathcal{H}_1$, then we have:

\begin{align}
        \mathcal{C}_1 &= \{x_5x_6,x_4x_6,x_2x_5,x_1x_4,x_2x_3,x_1x_2,x_1x_3x_6,x_3x_4x_5,x_1x_3x_5\}\\
        \mathcal{H}_1 &= \{x_5x_6,x_4x_6,x_2x_5,x_1x_4,x_2x_3,x_1x_2,x_1x_3x_6,x_3x_4x_5,x_1x_3x_5\}
\end{align}

Notice that since there are some bases in the second level comes before the first level bases in the ordered list $\mathcal{B}$, we actually to first apply the algorithm on them before we apply the algorithm on the larger first-level bases. For example when we process the basis $x_5e_{1,2}$, the $\mathcal{C}_1$ we are using is $\{x_5x_6,x_4x_6,x_2x_5,x_1x_4,x_2x_3,x_1x_2\}$. However in this example, we don't have any bases in the second level that cannot be vanished using the smaller bases in the first-level. Similarly, we also have some bases from the third level appears before the second-level bases in the list. They are 
\begin{equation}
    x_5e_{2,6},\;x_5e_{2,7},\;x_4e_{2,7}
\end{equation}

This is where we need to be especially careful with the order in the list, since they are using different $\mathcal{C}_2$ when they are compute. For $x_5e_{2,6}$, we have

\begin{equation}
    \begin{aligned}
        \mathcal{C}_2 = \{&x_5e_{1,2}-x_4e_{1,1},x_6e_{1,3}-x_2e_{1,1},x_6e_{1,4}-x_1e_{1,2},x_5e_{1,5}-x_3e_{1,3},x_4e_{1,6}-x_2e_{1,4},\\
        & x_3e_{1,6}-x_1e_{1,5},x_6e_{1,8}-x_3x_4e_{1,1},x_5e_{1,7}-x_1x_3e_{1,1},x_6e_{1,9}-x_1x_3e_{1,1},\\
        & x_4x_6e_{1,5}-x_2x_3e_{1,2},x_4e_{1,7}-x_1x_3e_{1,2},x_2e_{1,7}-x_1x_6e_{1,5},x_2e_{1,8}-x_4x_5e_{1,5},\\
        & x_1e_{1,8}-x_3x_5e_{1,4},x_4e_{1,9}-x_3x_4e_{1,4},x_2x_5e_{1,4}-x_4x_5e_{1,6}\}
    \end{aligned}
\end{equation}

Applying \verb+Reduce+ on $x_5e_{2,6}$ and the corresponding $\mathcal{C}_2$, we have 
    
\section{Conclusion}

\paragraph{  }

\textcolor{BrickRed}{waiting for editing}

\newpage
\printbibliography

\end{document}