\documentclass{article}

% Language setting
% Replace `english' with e.g. `spanish' to change the document language
\usepackage[english]{babel}

% Set page size and margins
% Replace `letterpaper' with`a4paper' for UK/EU standard size
\usepackage[letterpaper,top=2cm,bottom=2cm,left=3cm,right=3cm,marginparwidth=1.75cm]{geometry}

% Useful packageshttps://www.overleaf.com/project/617db8234c85bd8404625de7
\usepackage{amsmath}
\usepackage{graphicx}
\usepackage{amsthm}
\usepackage{amssymb}
\usepackage[colorlinks=true, allcolors=blue]{hyperref}
\usepackage[dvipsnames]{xcolor}
\newtheorem{theorem}{Theorem}[section]
\newtheorem{corollary}{Corollary}[theorem]
\newtheorem{lemma}[theorem]{Lemma}
\usepackage{float}
\usepackage[dvipsnames]{xcolor}
\usepackage{algorithm}
\usepackage{algpseudocode}
\usepackage[all]{xy}

\newcommand{\im}{\ensuremath{\operatorname{im}}}
\newcommand{\lm}{\ensuremath{\operatorname{LM}}}
\newcommand{\lt}{\ensuremath{\operatorname{LT}}}
\newcommand{\lc}{\ensuremath{\operatorname{LC}}}
\newcommand{\coker}{\ensuremath{\operatorname{coker}}}
\newcommand{\initTerm}{\ensuremath{\operatorname{in}}}
\newcommand{\mingens}{\ensuremath{\operatorname{mingens}}}
\newcommand{\lex}{\ensuremath{\operatorname{lex}}}
\newcommand{\grlex}{\ensuremath{\operatorname{grlex}}}
\newcommand{\grevlex}{\ensuremath{\operatorname{grevlex}}}
\newcommand{\multideg}{\ensuremath{\operatorname{multideg}}}
\renewcommand{\to}{\ensuremath{\rightarrow}}
\newcommand{\onto}{\ensuremath{\twoheadrightarrow}}
\newcommand{\into}{\ensuremath{\hookrightarrow}}

\usepackage[
backend=biber,
style=alphabetic,
sorting=ynt
]{biblatex}

\usepackage{alltt}

\addbibresource{research.bib}

\theoremstyle{definition}
\newtheorem{definition}{Definition}[section]

\theoremstyle{remark}
\newtheorem*{remark}{\textbf{Remark}}

\theoremstyle{example}
\newtheorem{example}{\textbf{Example}}[section]

\title{Computing and Minimalizing Schreyer Resolution\\
\begin{large} 
  Senior Project
\end{large}}
\author{Ruiqi (Rickey) Huang}

\newcommand{\qedwhite}{\hfill \ensuremath{\Box}}

\newtheorem{prop}{Proposition}[section]

\begin{document}
\maketitle

\begin{abstract}
    \textcolor{BrickRed}{waiting for editing}
\end{abstract}

\section{Introduction}

\paragraph{  }

Schreyer resolution of a module is an exact sequence of modules, and computing the Schreyer resolution of a module is an essential way to understand the invariant of the module. This paper investigates an algorithm for computing and minimalizing a Schreyer resolution of a free module found by La Scala and Stillman \cite{la_scala_strategies_1998}. The algorithm gives a minimal resolution in the graded case and gives the Betti numbers of the minimal resolution for free.

\section{Preliminaries}

\paragraph{  }

Before discussing the Schreyer resolution, the definition of Gröbner bases and syzygies should first be introduced, since they are the primary building blocks of the resolution. The Gröbner bases and syzygies are of importance to build the differentials connecting the free modules.

\subsection{Gröbner Basis}

Since we seek for a basis for the polynomial ring with multiple variables, we want to define terms such as leading terms and leading monomials, so we need to understand the ordering of the monomials in $k[x_1,\cdots,x_n]$ for some field $k$.

\begin{definition}[\textbf{Degree}]
    Let $\gamma = (\gamma_1,\cdots, \gamma_n)$ be a vector in $\mathbb{Z}_{\geq 0}^n$, then $x^{\gamma} \in k[x_1,\cdots,x_n]$ represents $x_1^{\gamma_1}x_2^{\gamma_2}\cdots x_n^{\gamma_n}$. Then, we define the \textbf{degree} of a polynomial to be the sum of all components of the variables. For example, the degree of $x^{\gamma}$ is $\lvert (\gamma_1, \cdots, \gamma_n) \rvert = \gamma_1 + \cdots + \gamma_n$.
\end{definition}

\begin{definition}
    For the polynomial ring $k[x_1,\cdots, x_n]$ with some field $k$, the order one the variables is defined as
    \begin{equation}
        x_1 > x_2 > \cdots > x_n
    \end{equation}
\end{definition}

By convention, we always write the variables in a monomial and the monomials in a polynomial in descending order.  Next, we introduce three monomial term orders in the following.

\begin{definition}[\textbf{Lexicographic (Lex) Order}]
    Let $\alpha = (\alpha_1, \cdots, \alpha_n)$ and $\beta = (\beta_1,\cdots,\beta_n)$ be in $\mathbb{Z}_{\geq 0}^{n}$. We say $\alpha >_{\lex} \beta$ if the leftmost nonzero entry of the vector difference $\alpha - \beta \in \mathbb{Z}^n$ is positive. We will write $x^{\alpha} >_{\lex} x^{\beta}$ if $\alpha >_{\lex} \beta$.
\end{definition}

\begin{definition}[\textbf{Graded Lex (GrLex) Order}]
    Let $\alpha,\beta \in \mathbb{Z}_{\geq 0}^{n}$. We say $\alpha >_{\grlex} \beta$ if
    \begin{equation}
        \lvert \alpha \rvert = \sum_{i = 1}^{n}{\alpha_i} > \lvert \beta \rvert = \sum_{i = 1}^{n}{\beta_i} \text{, or } \lvert \alpha \rvert = \lvert \beta \rvert \text{ and } \alpha >_{\lex} \beta
    \end{equation}
\end{definition}

\begin{example}
    $x_{1}^2x_{2}^{5}x_{3} >_{\grlex} {x_{1}}x_3^{3}$ since for the coefficient tuples $\lvert (2,5,1) \rvert = 8 > \lvert (1,0,3) \rvert = 4$, then $(2,5,1) >_{\grlex} (1,0,3)$. 
    
    $x_1^5x_2x_3^2 >_{\grlex} x_1x_2^{3}x_3^4$ since for the coefficient tuple $\lvert (5,1,2) \rvert = \lvert (1,3,4) \rvert = 8$, and $(5,1,2) - (1,3,4) = (4,-2,-2)$, then $(5,1,2) >_{\grlex} (1,3,4)$.
\end{example}

\begin{definition}[\textbf{Graded Reverse Lex (DegRevLex) Order}]
    Let $\alpha,\beta \in \mathbb{Z}_{\geq 0}^{n}$. We say $\alpha >_{\grevlex} \beta$ if 
    \begin{equation}
        \lvert \alpha \rvert = \sum_{i = 1}^{n}{\alpha_i} > \lvert \beta \rvert = \sum_{i = 1}^{n}{\beta_i} \text{, or } \lvert \alpha \rvert = \lvert \beta \rvert \text{ and the rightmost nonzero entry of } \alpha  - \beta \in \mathbb{Z}^n \text{ is negative.}
    \end{equation}
\end{definition}

\begin{example}
    $x_{1}^2x_{2}^{5}x_{3} >_{\grevlex} x_{1}x_3^{3}$ since for the coefficient tuples $\lvert (2,5,1) \rvert = 8 > \lvert (1,0,3) \rvert = 4$, then $(2,5,1) >_{\grevlex} (1,0,3)$. 
    
    $x_1^5x_2x_3^2 >_{\grevlex} x_1x_2^{3}x_3^4$ since for the coefficient tuple $\lvert (5,1,2) \rvert = \lvert (1,3,4) \rvert = 8$, and $(5,1,2) - (1,3,4) = (4,-2,-2)$, then $(5,1,2) >_{\grevlex} (1,3,4)$.
\end{example}

We choose to use the \verb+DegRevLex+ order for anywhere need the monomial term order in a polynomial ring in this paper because of the nice properties of the \verb+DegRevLex+ order for computer to do the computation. With the well-defined monomial ordering, we could define the basis with a nice property in describe the polynomial in an ideal: the Gröbner basis.

\begin{definition}[\textbf{Gröbner Basis}]\cite{cox_grobner_2015}\label{def:gb}
    Fix a monomial order on the polynomial ring $k[x_1, \cdots, x_n]$. A finite subset $G = \{g_1, \cdots, g_t\}$ of an ideal $I \subseteq k[x_1, \cdots, x_n]$ different from $\{0\}$ is said to be \textbf{Gröbner basis} (or \textbf{standard basis}) of $I$ if
    \begin{center}
        $\langle \lt(g_1), \cdots, \lt(g_t)\rangle = \langle \lt(I) \rangle$
    \end{center}
\end{definition}

%\begin{theorem}[\textbf{Hilbert Basis Theorem}]\cite{cox_grobner_2015}\label{thm:hb_thm}
 %   Every ideal $I \subseteq k[x_1,\cdots,x_n]$ has a finite generating set. In the other words, $I = <g_1,\cdots,g_t>$, for some $g_1,\cdots,g_t \in I$.
%\end{theorem}

%Theorem \ref{thm:hb_thm} implies that every ideal $I \subseteq k[x_1, \cdots, x_n]$ has a finite generating set. In other words, $I = <g_1, \cdots, g_t>$ for some $g_1, \cdots, g_t \in I$. Thus, we have the following corollary.

%\begin{corollary}\cite{cox_grobner_2015}\label{cor:gbI}
 %   Fix a monomial order on the polynomial ring $k[x_1,\cdots,x_n]$,every ideal in $k[x_1, \cdots, x_n]$ has a Gröbner basis. Additionally, any Gröbner basis for an ideal $I$ is a basis of $I$.
%\end{corollary}

\paragraph{}

Then, since $\langle LT(I)\rangle$ clearly contains $\langle \lt(g_1), \cdots, \lt(g_s) \rangle$, this set $G$ is a Gröbner basis of $I$ if $\langle \lt(I) \rangle \subseteq \langle \lt(g_1),\cdots, \lt(g_s) \rangle$, that is, the leading term of all elements of the ideal can be written as a $k[x_1, \cdots, x_n]$-linear combination of elements in $\langle \lt(g_1), \cdots, \lt(g_s) \rangle$. Furthermore, there are some properties of Gröbner basis as the followings.

\begin{itemize}
    \item Dividing $f$ by a Gröbner basis ($G = \{g_1, g_2, \cdots, g_t\}$), the remainder would be unique, but the quotient $q = \{q_1,\cdots,q_t\}$ would be different if the elements of $G$ are listed in a different order.
    \item $f \in I$ if and only if $0$ is a remainder of $f$ on division by $G$
    \item Let $\overline{f}^{F}$ is the remainder on division of $f$ by the ordered s-tuple $F = (f_1, f_2, \cdots, f_s)$, then $\overline{f}^{G} = 0$ if and only if $f \in I$
\end{itemize}

The uniqueness of the remainder can easily be checked by the division algorithm. Consider an example like $f = xy \in I \in \mathbb{Q}[x,y,z]$, and let $G_1 = \{x+z,y-z\}$ and $G_2 = \{y-z,x+z\}$ be a Gröbner basis written using two different orderings; then by the division algorithm one has

\begin{align}
        \text{For } G_1 \text{: } f &= y\cdot(x+z) - z\cdot (y-z) - z^2\\
        \text{For } G_2 \text{: } f &= x\cdot(y-z) + z\cdot (x+z) - z^2
\end{align}

One could see that even though the remainders are the same, the quotients are different with two different bases. The second and third properties could be proved following the definition of Gröbner basis, since $\langle \lt(I) \rangle \subseteq \langle \lt(g_1),\cdots, \lt(g_s) \rangle$.


\begin{remark}
    An obstruction for a basis being a Gröbner basis is that the cancellation occurs in $f = ax^{\alpha}g_i - bx^{\beta}g_{j}$, where $g_i,g_j \in I$, and $f \in \langle \lt(I) \rangle$, but $f \notin \langle \lt(g_i), \lt(g_j)\rangle$.
\end{remark}

\paragraph{}

For the convenience and efficiency of the computation, we always want our Gröbner basis to be as small as possible while it still has the same properties introduced above. Thus, we could make the following two definitions of minimal and reduced Gröbner basis.

\begin{definition}[\textbf{Minimal Gröbner Basis}]
    The \textbf{minimal} Gröbner basis is a Gröbner basis  $G$ for $I$ such that:
    \begin{itemize}
        \item $\lc(p) = 1$ for all $p \in G$
        \item For all $p \in G$, none of $\lt(p)$ lies in $\langle \lt(G\backslash\{p\}) \rangle$
    \end{itemize}
\end{definition}

\begin{definition}[\textbf{Reduced Gröbner Basis}]
    The \textbf{reduced} Gröbner basis is a Gröbner basis $G$ for $I$ such that:
    \begin{itemize}
        \item $\lc(p) = 1$ for all $p \in G$
        \item For all $p \in G$, none of the monomials in $p$ lies in $\langle \lt(G\backslash\{p\})\rangle$
    \end{itemize}
\end{definition}

\begin{example}
    Let $\{xy+yz,yz+xyz\}$ be a Gröbner basis of $I \in \mathbb{Q}[x,y,z]$, $\{xy+yz,yz+xyz\}$ is a minimal Gröbner basis, but it is not a reduced Gröbner basis. If we further reduced it, we could obtain $\{xy,yz\}$, which is a reduced Gröbner basis.
\end{example}

\subsection{Syzygy}

Because Gröbner basis has such a pleasant property of describing polynomials with linear combination of basis elements and a unique remainder, it would be nice for one to know how to compute the Gröbner basis of an ideal in an efficient way,so we introduce Buchberger's algorithm. However, we first need some more definitions to understand the algorithm. First, we need to define the $S$-polynomial and syzygy.

\begin{definition}[\textbf{Least Common Multiple}]
    Let $f, g \in k[x_1, \cdots, x_n]$ be nonzero polynomials such that $multideg(f) = \alpha$ and $multideg(g) = \beta$, then let $\gamma = (\gamma_1, \cdots, \gamma_n)$, where $\gamma_i = max(\alpha_i,\beta_i)$. Then $x^{\gamma}$ is the least common multiple of $LM(f)$ and $LM(g)$, that is, $x^{\gamma} = lcm(\lm(f), \lm(g))$.
\end{definition}

\begin{definition}[\textbf{S-polynomial}]
    Let $f, g \in k[x_1, \cdots, x_n]$ be nonzero polynomials, and let $x^{\gamma} = lcm(LM(f), LM(g))$ then the S-polynomial of $f$ and $g$ is defined as a polynomial that cancels leading terms of both the polynomials. To be specific,
    \begin{equation}
        S(f,g) = \tfrac{x^{\gamma}}{\lt(f)} \cdot f - \tfrac{x^{\gamma}}{\lt(g)}\cdot g
    \end{equation}
\end{definition}

\paragraph{  }
We can easily check the cancelation of the leading terms simply by the following calculation:

Let $f' = f - LT(f)$ and $g' = g - LT(g)$. Then:
\begin{align}
    \tfrac{x^{\gamma}}{\lt(f)} \cdot f - \tfrac{x^{\gamma}}{\lt(g)} \cdot g &= \tfrac{x^{\gamma}}{(\lt(f) + f')} - \tfrac{x^{\gamma}}{(\lt(g) + g')}\\
    &= x^{\gamma} + \tfrac{x^{\gamma}}{\lt(f)} \cdot f' - x^{\gamma} - \tfrac{x^{\gamma}}{\lt(g)} \cdot g'\\
    &= \tfrac{x^{\gamma}}{\lt(f)} \cdot f' - \tfrac{x^{\gamma}}{\lt(g)} \cdot g' 
\end{align}

As what $\tfrac{x^{\gamma}}{\lt(f)}$ and $\tfrac{x^{\gamma}}{\lt(g)}$ do in the previous calculation, they help to vanish the leading terms of the polynomials with performing the "dot product". To generalize this process, we define such monomials for any finite set of polynomials as the syzygy of this set of polynomials.

\begin{definition}[\textbf{Syzygy}]
    Let $F = (f_1, \cdots, f_l) \in {(k[x_1,\cdots,x_n]})^l$. A \textbf{syzygy} on the leading terms $\lt(f_1), \cdots, \lt(f_l)$ of $F$ is a $l$-tuple of polynomials $S = (s_1,,\cdots, s_l) \in {(k[x_1,\cdots,x_n]})^l$ such that
    \begin{equation}
        \sum_{i = 1}^{l} s_i \cdot \lt(f_i) = 0
    \end{equation}
\end{definition}

\begin{example}
    Let $F = (3xy^2 + y, xyz+x^2z, 5y^3+7xyz)$, then $S = (yz,2y^2,-xz)$ is a syzygy of $F$, because $3xy^2\cdot yz +xyz \cdot 2y^2 + 5y^3\cdot(-xz) = 3xy^3z + 2xy^3z - 5xy^3z = 0$.
\end{example}

%\begin{lemma}
 %   For a sum $\sum_{i = 1}^{s}{p_i}$, where $multideg(p_i) = \delta \in Z_{\geq0}^{n}$ for all $i$, if $multideg(\sum_{i = i}^{s}{p_i}) < \delta$, then $\sum_{i = 1}^{s}{p_i}$ is a linear combination, with coefficients in $k$, of the S-polynomials $S(p_j, p_l)$ for $1 \leq j,l \leq s$. Furthermore, each $S(p_j, p_l)$ has multidegree $< \delta$.
%\end{lemma}

%\textcolor{BrickRed}{include the proof later}

\subsection{Buchberger's Algorithm}

\paragraph{}

Now, we could outline the details of the Buchberger algorithm.

\begin{theorem}[\textbf{Buchberger's Algorithm}]
    Let $I = \langle f_1,\cdots,f_s \rangle \neq \{0\}$ be a polynomial ideal. Then a Gröbner basis for I can be constructed in a finite number of steps by the following algorithm:
    \begin{algorithm}[H]
        \caption{Buchberger's Algorithm}\label{alg:bc}
        \begin{algorithmic}
        \State Input: F = $(f_{1},\cdots,f_{s})$
        \State Output: a Gröbner basis $G=(g_{1},\cdots,g_{t})$ for $I$, with $F \subseteq G$
        \newline
        \State $G:=F$
        \State REPEAT
            \State $G':=G$
            \For{each pair$\{p,q\}, p \neq q$ in $G'$}
                \State $r := {\overline{S(p,q)}}^{G'}$
                \If{$r\neq 0$}
                    \State $G\:=G\cup\{r\}$
                \EndIf
            \EndFor
        \State UNTIL $G=G'$
        \State return $G$
    \end{algorithmic}
    \end{algorithm}
\end{theorem}

\begin{proof}

To prove the algorithm, we first need to understand the connection between $S$-polynomials and Gröbner bases by the theorem below.

\begin{theorem}[Buchberger's Criterion]\label{thm:buchCriter}
    Let $I$ be a polynomial ideal. Then a basis $G = \{g_1, \cdots, g_t\}$ of $I$ is a Gröbner basis of $I$ if and only if $\forall \{i,j\}$ such that $i \neq j$, the remainder on the division of $S(g_i,g_j)$ by $G$ is zero.
\end{theorem}

The proof of Theorem \ref{thm:buchCriter} could be referred to the proof of Theorem 6 of Chapter $2 \;\S6$ in Cox, Little, and O'Shea's book \cite{cox_grobner_2015}. 

Because for any pair of $p,q$, $S(p,q) = \sum_{i = 1}^{t}{h_ig_i} + r$ for $g_i \in G'$ with $i = 1, \cdots, t$. Then we could show that $S(p,q) \in I$ since $p,q \in G' \subseteq I$ and the absorption property of $I$ guarantee that the two polynomials summed when computing $S(p,q)$ are in $I$. Also by the absorption property, since $g_i \in G' \subseteq I$, each $h_ig_i \in I$. Therefore, $\sum_{i = 1}^{t}{h_ig_i} \in I$ and $r = S(p,q) -  \sum_{i = 1}^{t}{h_ig_i} \in I$. Therefore, for each iteration, $G\:=G\cup\{r\} \subseteq I$, so $G$ is a basis for $I$. Since the algorithm above ends when the $G = G'$, that is, when the remainder $r = 0, \; \forall p \neq q \in G'$, applying Theorem \ref{thm:buchCriter}, $G$ is a Gröbner basis of $I$ at the end of the algorithm.

It is sufficient to show that the algorithm ends in finite steps. To prove that the algorithm terminates, we need the Ascending Chain Condition as follows.

\begin{theorem}[The Ascending Chain Condition (ACC)]
    Let $I_1 \subseteq I_2 \subseteq I_3 \subseteq \cdots$ be an ascending chain of ideals in $k[x_1, \cdots, x_n]$. Then there exists an $N \geq 1$ such that $I_{N} = I_{N+1} = I_{N+2} = \cdots$.
\end{theorem}

The proof of Theorem \ref{thm:buchCriter} could be referred to the proof of Theorem 7 of Chapter $2 \;\S5$ of Cox, Little, and O'Shea's book \cite{cox_grobner_2015}. 

Since for each iteration, we have $G' \subseteq G$, then $\langle \lt(G') \rangle \subseteq \langle \lt(G) \rangle$. Let $G_0 = F$, and if we keep track of $G$ from each iteration and label them as $G_1, G_2, \cdots$, then we have an ascending chain of ideals $\langle \lt(G_0) \rangle \subseteq \langle \lt(G_1) \rangle \subseteq \langle \lt(G_2) \rangle \subseteq \cdots$. Therefore, applying ACC, we know that $\exists N \geq 1$ such that $\langle \lt(G_{N}) \rangle = \langle \lt(G_{N+1}) \rangle = \langle \lt(G_{N+2}) \rangle = \cdots$. This shows that the algorithm ends at some point with a finite number of steps.
\end{proof}

\begin{remark}
    The Buchberger's algorithm only produces a Gröbner Basis which is not unique, and it might include redundancy. The Gröbner basis produced by the algorithm is not necessarily a reduced Gröbner basis, since we only check the leading terms of the polynomials using the $S$-polynomials instead of checking for every monomial in the polynomials. Moreover, it even does not promise a minimal Gröbner basis. For example, when we input Algorithm \ref{alg:bc} with $F$, which is already a Gröbner basis with redundancy, which means that there is at least one polynomial $p \in F$ such that $\lt(p) \in \langle \lt(F\backslash \{p\}) \rangle$, the algorithm would output $G = F$ directly by Theorem \ref{thm:buchCriter}, which is not minimal by definition.
\end{remark}

\subsection{Vector Space Spanned by the leading monomials}

\paragraph{}

In the next definition, we define a way to build vector spaces using subsets of the ideal $S$.

\begin{definition}
    Let $S = k[x_1,\cdots, x_n]$ be a polynomial ring with a term order $>$ over a field $k$. For any set $G \subseteq S$, we let $\initTerm(G)$ denote the \textbf{$k$-vector space spanned by the monomials} $\{\lm(f):f \in G\}$.
\end{definition}

\paragraph{}

Because $R = k[x_1,\cdots,x_n]$, the scalar multiples of $in(G)$ are well defined. Since the monomials of $G$ are in $R$ and $R$ is a polynomial ring, $in(G)$ indeed defines a $k$-vector space, and this is indeed a monomial ideal generated by the leading terms of elements in $G$.

\begin{prop}
    Let $S = k[x_1,\cdots, x_n]$ be a polynomial ring with a term order $>$ over a field $k$, $J \subset S$ and ideal and $R = S/J$ be a factor ring of $S$. Let $N$ denote the $k$-vector space spanned by the set of standard monomials of $R$, that is, the set of monomials of $S$ not in $in(J)$. Any element $f \in R$ can be written uniquely as the image of an element $g \in N$, and we set $lc(f) = lc(g)$ and $lm(f) = lm(g) \in N$. Thus, $N = \initTerm(R)$ and $in(S) = N \bigoplus \initTerm(J)$, as $k$-vector spaces.
\end{prop}

\begin{proof}
    Let $m \in \initTerm(S)$, then let $m = \sum_{i = 1}^{l}{q_i\cdot \lm(s_i)}$ for $q_i \in k$ and $s_i \in S$. Suppose that in $\{\lm(s_1),\cdots,\lm(s_l)\}$, there are $p$ of them in $\initTerm(J)$. Then, by relabeling $s_i's$, we have $m = \sum_{i = 1}^{p}{q_i \cdot \lm(s_i)} + \sum_{j = p+1}^{l}{q_j\cdot \lm(s_j)}$ for $\lm(s_i) \in \initTerm(J)$ and $\lm(s_j) \in S\backslash J$. Let $u = \sum_{i = 1}^{p}{q_i \cdot \lm(s_i)}$ and $v = \sum_{j = p+1}^{l}{q_j\cdot \lm(s_j)}$. Then, by definition, $u \in \initTerm(J)$, $v \in N$, and $m = u + v$. Since we choose $m$ arbitrarily and $u,v$ to be unique by construction, we know that $in(S) = N \bigoplus \initTerm(J)$.
\end{proof}

\section{Schreyer Frame and the Corresponding Schreyer Resolution}

\paragraph{}

Before introducing the Schreyer resolutions and their frames, we need to first define the following terms in rings, and modules.

\begin{definition}[\textbf{Power Product}]
    Let $S = k[x_1,\cdots, x_n]$ be a polynomial ring with a term order $>$ over a field $k$. A \textbf{power product} $t$ (or monomial of $S$) is $t = x_1^{\alpha_1}\cdots x_n^{\alpha_n} \in S$, where $\alpha_1, \cdots, \alpha_n$ are non-negative integers.
\end{definition}

With this definition, we could uniquely write every nonzero element $f \in S$ as 
\begin{equation}
    f = c_1\cdot m_1 + \cdots + c_l\cdot m_l
\end{equation}
where for $i = 1, \cdots, l$, $c_i \in k$ and are nonzero, and $m_i$ are monomials such that $m_1 > m_2 > \cdots > m_l$. In addition, we define the leading coefficient of $f \in S$ to be $lc(f) = c_1$, and the leading monomial of $f$, $\lm(f) = m_1$.

\subsection{Free Module over A Factor Ring}

\paragraph{}

With such a factor ring, we could build a free module from the ring, and we define the elements and the term orders on the elements of the newly constructed module $F$ as follows.

\begin{definition}[\textbf{Monomials of a Module}]
    Let $F$ be a free module over $R$, and let $\hat{F}$ be the free $S$-module with the same rank as $F$ and corresponding basis. A \textbf{monomial} of $F$ is, by definition, any element $m = t \cdot e$, where $t \in N$ is a standard monomial and $e$ is any element of the canonical basis of $\hat{F}$.
\end{definition}

\begin{definition}[\textbf{Term order on a Free Module}]
    The term order on a free module $F$ over $k$ (a total order on the monomials of $F$) such that:
    \begin{itemize}
        \item if $m <_F n$, then $t\cdot m <_F t \cdot n$;
        \item if $s <_S t$, then $s\cdot e <_F t \cdot e$;
    \end{itemize}
    for all $m, n$ monomials of $F$, $s,t$ power products in $S$, and $e$ any basis element of $F$.
\end{definition}

\paragraph{}

Using the $k$-vector space spanned by leading monomials of elements of a subset of and ideal, we could define the Gröbner basis of a free module in a different way. We define that $\{g_1, \cdots, g_s\} \subset I\subset F$ is a Gröbner basis of the $R$-module $I$, if $\{lm(g_1), \cdots, lm(g_s)\}$ generates $in(I)$, that is, every monomial in $in(I)$ is divisible by some $lm(g_i)$.

\subsection{Free Resolution}

As the main structure of the resolutions and frames, complex and exact sequences must first be defined.

\begin{definition}[\textbf{Complex and Exactness}]
    Let $\Phi = \{\varphi_n : A_n \to A_{n-1}\}$ be a sequence of homomorphisms of $R$-modules for $n \in \mathbb{Z}$: 
    \begin{equation}
        \Phi : \cdots \to A_{n+1} \xrightarrow{\varphi_{n+1}} A_n \xrightarrow{\varphi_n} A_{n-1} \to \cdots 
    \end{equation}
    Then, $\Phi$ is a \textbf{complex} if $\forall n \in \mathbb{Z}$, $\im \varphi_{n+1} \subseteq \ker \alpha_n$.  If $\im \varphi_{i+1} = \ker \varphi_i$, $\Phi$ is \textbf{exact} in (homological) degree $i$.  If $\Phi$ is exact
    in every homological degree, we say that $\Phi$ is exact.  If $\Phi$ is exact at every homological degree except 0, then we say $\Phi$ is \textbf{acyclic}.
\end{definition}

Then, we could introduce the definition of the free resolution.

\begin{definition}[\textbf{Free Resolution}]
    Let $R = S/J$ be the factor ring as previously defined, and let $F_i$ be free $R$-modules with a given (canonical) basis $\mathcal{E}_{i}$ of $\hat{F_i}$ and $M = F_0/I$.  A \textbf{free resolution} of $M$ is an acyclic complex of free $R$-modules:

    \begin{equation}
        \Phi: \cdots \rightarrow F_l \xrightarrow{\varphi_l} F_{l-1} \xrightarrow{\varphi_{l-1}} \cdots \xrightarrow{\varphi_1} F_0
    \end{equation}
    
    Define $\mathcal{C}_i = \varphi_i(\mathcal{E}_i)$ as the image of the given basis and define the level of an element $f$ in $C_i$ as $lev(f) = i$.
\end{definition}

\begin{prop}
    To (theoretically) construct a free resolution, we first choose a (finite) generating set of $M$, which gives a surjection $F_0 \onto M$ where $F_0$ is a finite rank free $R$-module. The kernel of this map $K_1$ is again a finitely generated $R$-module because $R$ is Noetherian. So we get a short exact sequence 
    \begin{equation}
        0 \into K_1 \into F_0 \onto M \to 0
    \end{equation}
    where $K_1$ is a finitely generated $R$-module.  We may now repeat this process: choose a finite generating set of $K_1$, which gives an onto map $\varphi_1': F_1 \to K_1$, where $F_0$ is again a finitely rank free $R$-module, which in turn gives rise to a short exact sequence
    \begin{equation}
        0 \into K_2 \into F_1 \onto K_1 \to 0.
    \end{equation}
    where $K_2$ is the kernel of the map Also, since the composition $\varphi_1: F_1 \to F_0$ is a homomorphism between finite rank free $R$-modules, it may be represented by a matrix of finite size. Continuing this process, we fit all these short exact sequences into an exact complex of modules as in the following diagram, and this is a free resolution of $M$.
    \begin{equation}\label{eqn:16}
        \xymatrix@C=1pc@R=1pc{
            &            &       &       &       &      &       &     &K_2 \ar@{^{(}->}[dr]^{\iota_2}\\
            &       &\cdots \ar[r]       &F_{l} \ar@{->>}[dr]_{\varphi_{l}'} \ar[rr]^{\varphi_{l}}&       &F_{l-1} \ar[r]^{\varphi_{l-1}}&\cdots \ar[r]^{\varphi_3}&F_2 \ar@{->>}[ur]^{\varphi_2'} \ar[rr]^{\varphi_2}&       &F_1 \ar@{->>}[dr]_{\varphi_1'} \ar[rr]^{\varphi_1} &         &  F_0 \ar@{->>}[r] & M \ar[r] & 0\\
            &        &       &       & K_{l} \ar@{^{(}->}[ur]_{\iota_{l}}        &        &        &      &       &       & K_1 \ar@{^{(}->}[ur]_{\iota_1}  &       &       & \\
        }
\end{equation}
\end{prop}

\begin{proof}
    To prove that this construction gives us a free resolution, we need to show that this sequence is exact at every homological degree except $0$. Since by construction, $K_2$ is the kernel of the map $\varphi_1: F_1 \onto F_0$, then $K_2 \subseteq F_1$ and $\iota_2: K_2 \into F_1$ is an inclusion. Therefore, the composition $\varphi_2 = \iota_2 \circ \varphi_2' = \varphi_2'$ satisfies $\im \varphi_2 = \im \varphi_2' = K_2 = \ker \varphi_1$ since $\varphi_2'$ is an onto map sending to $K_2$. Thus, the sequence is exact at degree $1$. Suppose the sequence is exact at degree $t$, then $F_t \to K_t$ must defines an onto map, and by construction, we have the short exact sequence, 
    \begin{equation}
        0 \into K_{t+1} \into F_t \onto K_t \to 0
    \end{equation}
    where $K_{t+1}, F_t,$ and $K_t$ are all finitely generated free $R$-modules, and $K_{t+1}$ is the kernel of the map $F_{t} \onto K_t$. Next, we could construct a finitely generated free $R$-module $F_{t+1}$ such that there is an onto map $\varphi_{k+1}: F_{t+1} \onto K_{t+1}$. Then the function composition gives us $\varphi_{t+1} = \iota_{t+1} \circ \varphi_{t+1}' = \varphi_{t+1}'$, and $\varphi_{t+1}$ sends $F_{t+1}$ to $F_t$. Hence, $\im \varphi_{t+1} = \im \varphi_{t+1}' = K_{t+1} = \ker \varphi_{t}$. Therefore, the sequence is exact at level $t+1$. Thus, by induction, we know that whole sequence is an acyclic complex, so it is a free resolution.
\end{proof}

\paragraph{}

When we are trying to practically compute the free resolution following the construction above, one might encounter a problem of computing the kernels of matrix over a polynomial ring, which. in this case, a factor ring $R$. This is why the technique of Gröbner basis are useful here. To be more specific, we could first find the Gröbner basis of each $K_i$ ("row reduce" the matrix over the polynomial ring, i.e. finding the differential) and then compute a ordered set of polynomials that are multiplied with the basis and summed to zero like what a syzygy does (find a "vector" in the "null space" of the matrix), which is the kernel we want.

\begin{remark}
    In the construction above only defines ordering on each free $R$-modules $F_i$, there is not a well-defined ordering on the whole resolution. Thus, we define a term ordering on the free resolution.
\end{remark}

\begin{definition}[\textbf{Term Ordering on Free Resolution}]
    Let $\tau = \{\tau_i\}$ be a sequence of term ordering on the modules $F_i$. We call $\tau$ a \textbf{term ordering on $\Phi$} if it satisfies the following compatibility relationship:
    \begin{equation}
        s \cdot e_1 <_{\tau_i} t\cdot e_2 \text{ whenever } s \cdot \lm(\varphi_i(e_1)) <_{\tau_{i-1}} t \cdot \lm(\varphi_i(e_2))
    \end{equation}
    where $e_1$ and $e_2$ are elements of $\mathcal{E}_i$.
\end{definition}

To describe the resolution $\Phi$, we define the initial terms of $\Phi$, $\Xi$ as follows.

\begin{definition}[\textbf{Initial Term of A Resolution}]
    Given $\Phi$ as previously described and a term ordering $\tau$ on $\Phi$, define the \textbf{initial terms} of $\Phi$, $\initTerm(\Phi)$, as the sequence of (graded) $R$-homomorphisms:
    \begin{equation}
        \Xi = \initTerm(\Phi): \cdots \to F_l \xrightarrow{\xi_l} F_{l-1} \xrightarrow{\xi_{l-1}} \cdots \xrightarrow{\xi_2} F_1 \xrightarrow{\xi_1} F_0
    \end{equation}
    where $\xi_i(e) = \lm(\varphi_i(e))$, for all $e$ in $\mathcal{E}_i$. That is, the differentials in $\initTerm(\Phi)$ are obtained by taking the leading monomials of each entry in the differentials in $\Phi$.
\end{definition}

\begin{remark}
    Since the homomorphisms defined in $\Xi$ are $\xi_i(e) = \lm(\varphi_i(e))$, the compatibility relationship is satisfied.
    \begin{equation}
        \begin{aligned}
            \lm(s) \cdot e_1 <_{\tau_i} \lm(t) \cdot e_2 \text{ whenever } & \lm(s) \cdot \lm(\xi_i(e_1)) <_{\tau_{i-1}} \lm(t) \cdot \lm(\xi_i(e_2))\\
                                                                           &\Leftrightarrow \lm(s) \cdot \lm(\lm(\varphi_i(e_1))) <_{\tau_{i-1}} \lm(t) \cdot \lm(\lm(\varphi_i(e_2)))\\
                                                                           &\Leftrightarrow \lm(s) \cdot \lm(\varphi_i(e_1)) <_{\tau_{i-1}} \lm(t) \cdot \lm(\varphi_i(e_2))
        \end{aligned}
    \end{equation}
    where $e_1$ and $e_2$ are elements of $\mathcal{E}_i$. Thus, the term ordering of $\Xi$ is also $\tau$.
\end{remark}

In order to practically compute a free resolution for a free module over a polynomial ring, we need to use the Gröbner basis, and if we use the Gröbner basis to compute the kernel connecting the whole resolution, we are actually creating a Schreyer resolution. We define the Schreyer resolution as the following. A Schreyer resolution is a free resolution with some restrictions.

\begin{definition}[\textbf{Schreyer Resolution}]
    A \textbf{Schreyer resolution} of an $R$-module $M = F_0/I$ is an exact sequence:
    \begin{equation}
        \Phi: \cdots \rightarrow F_l \xrightarrow{\varphi_l} F_{l-1} \xrightarrow{\varphi_{l-1}} \cdots \xrightarrow{\varphi_1} F_0
    \end{equation}
    , such that:
    \begin{itemize}
        \item $\coker(\varphi_1) = M$
        \item There is a term ordering $\tau$ on $\Phi$
        \item $\varphi_i(\mathcal{E}_i)$ forms a minimal Gröbner basis of $\im \varphi_i = \ker \varphi_{i-1}$ (for all i where $F_i \neq 0$)
    \end{itemize}
\end{definition}

\paragraph{}

Because a Schreyer resolution requires $\varphi_i(\mathcal{E}_i)$ to be a Gröbner basis, we want to focus on the initial term of the Schreyer resolution more. Therefore, we introduce the following definition of Schreyer frame, which is a similar term as the initial terms of a resolution in the Schreyer case.

\begin{definition}[\textbf{Schreyer Frame}]
    A \textbf{Schreyer frame} of $M = F_0/I$ is a sequence of $R$-homomophisms:
    \begin{equation}
        \Xi: \cdots \to F_1 \xrightarrow{\xi_l} F_{l-1} \xrightarrow{\xi_{l-1}} \cdots \xrightarrow{\xi_2} F_1 \xrightarrow{\xi_1} F_0
    \end{equation}
    where each column is a monomial, and a term ordering on $\Xi$ such that:
    \begin{itemize}
        \item $\xi_1(\mathcal{E}_1)$ is a minimal set of generators for $\initTerm(I)$;
        \item $\xi_i(\mathcal{E}_i)$ is a minimal set of generators for $\initTerm(\ker \xi_{i-1})$ for $i \geq 2$.
    \end{itemize}
\end{definition}
%With these properties of the Schreyer resolution, we can theoretically %compute the Schreyer resolution using the diagram below.

%We could find the kernel of homomorphism $F_0 \to M$, $K_0$. Because 
%$\coker(\varphi_1) = M$, $K_0 = \im \varphi_1$.

\paragraph{}

In this definition, we only trace how the leading monomials are changed in the resolution, where the homomorphisms $\xi_i(e_i) = \lm \varphi_i(e_i)$ for $e \in \mathcal{E}_1$. The restriction on the sets $\xi_i(\mathcal{E}_i)$ that it needs to be the minimal set such that $\xi_i(\mathcal{E}_i) = \initTerm(\ker \xi_{i-1})$ make sure that $\varphi_i(\mathcal{E}_i)$ forms a minimal Gröbner basis of $\ker \varphi_{i-1}$. From this construction, we could show that for any Schreyer resolution $\Phi$, $\initTerm (\Phi)$ is its corresponding Shreyer frame, and for any Shreyer frame $\Xi$, we could extend it to build a Shreyer resolution by including more information using the idea of syzygy.

With the definition of the Schreyer frame, we define some notations that would be useful when we prove the propositions later.

Let $e \in \mathcal{E}_{i-1}$, then
\begin{align}
    \mathcal{B}_i &= \xi_{i}(\mathcal{E}_i)\\
    \mathcal{E}_i(e) &= \{\epsilon \in \mathcal{E}_i \lvert \xi_{i}(\epsilon) = s \cdot e \text{, for some power product }s\}
\end{align}

%\textcolor{BrickRed}{How to practically find a free resolution: via the Schreyer frame}

\subsection{Computing Schreyer Frame}

\paragraph{}

Thus, we know that if we have a Schreyer frame, we are able to extend it into a Schreyer resolution, but we still need the following definition to build a Shreyer frame from a free module.

\begin{definition}[\textbf{Colon Ideal}]
    Let $I, J$ be two ideals of commutative rings R, then the \textbf{colon ideal} $(I:J)$ is defined as
    \begin{equation}
        (I:J) = \{r \in R \; \lvert \;rJ \subseteq I\}
    \end{equation}
    If $p$ is an polynomial in $R$, then 
    \begin{equation}
        (I:p) = \{r \in R \; \lvert \; rp \in I\}
    \end{equation}
\end{definition}

%\textcolor{BrickRed}{Remind definition of colon ideal}

Then, the next Lemma is crucial for us to build up a Schreyer frame, since we need to minimally generate $\initTerm (\ker (\xi_{i-1}))$ to find $\mathcal{B}_i.$
\begin{lemma}\label{lem:mingens}
    $in(ker\xi_i)$ is minimally generated by
    \begin{equation}
        \bigcup_{e\in{\mathcal{E}}_{i-1}}{\bigcup_{j = 2}^{r}{\mingens((in(J),t_1,\cdots,t_{j-1}):t_j)\cdot \epsilon_j}}
    \end{equation}
    where for each $e$ in the outer union, if  $\mathcal{E}_i(e) = \{\epsilon_1,\cdots,\epsilon_r\}$, then $\xi_i(\epsilon_j) = t_{j} \cdot e$, and $\mingens$ defines the subset of the minimal genreators of the considered monomial ideal which do not lie in $\initTerm(J)$.
    %\textcolor{BrickRed}{edit more}
\end{lemma}

The proof of this lemma could be referred to the proof of Lemma $3.5$ from the paper by La Scala and Stillman \cite{la_scala_strategies_1998}.

\begin{example}
    Let $K$ be a field of any characteristic and let the polynomial ring $S = K[x_1,\cdots,x_6]$ with the term order of \verb+DegRevLex+. Consider the module $M = S/I$, for 
    \begin{equation}
        I = \langle x_2x_3,x_2x_4,x_5x_6,x_1x_2,x_4x_6,x_2x_5,x_1x_3x_5,x_1x_3x_6,x_3x_4x_5\rangle
    \end{equation} 
    Using the algorithm \verb+Resolution+ for $char(K) \neq 2$ to the module $M$, consturct a Schreyer frame from the module $M$.
\end{example}

\paragraph{  }

\textbf{Solution:} We first want to compute the Schreyer frame $\Xi$ for $M$. With this plan, the basis of the first level of the Schreyer frame is $I$, because it generates the kernel of the map $F_0 \rightarrow M$. Hence, we have 
\begin{equation}
    \mathcal{B}_1 = \{x_2x_3,x_2x_4,x_5x_6,x_1x_2,x_4x_6,x_2x_5,x_1x_3x_5,x_1x_3x_6,x_3x_4x_5\}
\end{equation}
After ordering it using the \verb+DegRevLex+, the first level basis is:
\begin{equation}
    \mathcal{B}_1 = \{x_5x_6,x_4x_6,x_2x_5,x_1x_4,x_2x_3,x_1x_2,x_1x_3x_6,x_3x_4x_5,x_1x_3x_5\}
\end{equation}
Then, we used the formula in the Lemma \ref{lem:mingens}, to construct the basis of the next level by finding the monomials that knock a basis into the ideal once at a time.
\begin{align}
    j &= 2: (x_5x_6:x_4x_6) = \{x_5\} \rightarrow \{x_5e_{1,2}\}\\
    j &= 3: ((e_{1,1},e_{1,2}):x_2x_5) = \{x_6,x_4x_6\} = \{x_6\} \rightarrow \{x_6e_{1,3}\}\\
    j &= 4: ((e_{1,1},\cdots,e_{1,3}):x_1x_4) = \{x_5x_6,x_6,x_2x_5\} = \{x_6,x_2x_5\} \rightarrow \{x_6e_{1,4},x_2x_5e_{1,4}\}\\
    j &= 5: ((e_{1,1},\cdots,e_{1,4}):x_2x_3) = \{x_5x_6,x_4x_6,x_5,x_1x_4\} = \{x_5,x_4x_6,x_1x_4\}\\
    &\rightarrow \{x_5e_{1,5},x_4x_6e_{1,5},x_1x_4e_{1,5}\}\\
    j &= 6: ((e_{1,1},\cdots,e_{1,5}):x_1x_2) = \{x_5x_6,x_4x_6,x_5,x_4,x_3\} = \{x_5,x_4,x_3\}\\
    &\rightarrow \{x_5e_{1,6},x_4e_{1,6},x_3e_{1,6}\}\\
    j &= 7: ((e_{1,1},\cdots,e_{1,6}):x_1x_3x_6) = \{x_5,x_4,x_2x_5,x_4,x_2,x_2\} = \{x_5,x_4,x_2\}\\
    &\rightarrow \{x_5e_{1,7},x_4e_{1,7},x_2e_{1,7}\}\\
    j &= 8: ((e_{1,1},\cdots,e_{1,7}):x_3x_4x_5) = \{x_6,x_6,x_2,x_1,x_2,x_1x_2,x_1x_6\} = \{x_6,x_2,x_1\}\\
    &\rightarrow \{x_6e_{1,8},x_2e_{1,8},x_1e_{1,8}\}\\
    j &= 9: ((e_{1,1},\cdots,e_{1,8}):x_1x_3x_5) = \{x_6,x_4x_6,x_2,x_4,x_2,x_2,x_6,x_4\} = \{x_6,x_4,x_2\}\\
    &\rightarrow \{x_6e_{1,9},x_4e_{1,9},x_2e_{1,9}\}
\end{align}
Therefore ,we have our second level basis of $\Xi$:
\begin{equation}
    \begin{aligned}
        \mathcal{B}_2 = \{&x_5e_{1,2},x_6e_{1,3},x_6e_{1,4},x_2x_5e_{1,4},x_5e_{1,5},x_4x_6e_{1,5},x_1x_4e_{1,5},x_5e_{1,6},x_4e_{1,6},x_3e_{1,6},x_5e_{1,7},\\
        & x_4e_{1,7},x_2e_{1,7},x_6e_{1,8},x_2e_{1,8},x_1e_{1,8},x_6e_{1,9},x_4e_{1,9},x_2e_{1,9}\}
    \end{aligned}
\end{equation}
After ordering it using the \verb+DegRevLex+, the second level basis is:
\begin{equation}
    \begin{aligned}
        \mathcal{B}_2 = \{&x_5e_{1,2},x_6e_{1,3},x_6e_{1,4},x_5e_{1,5},x_5e_{1,6},x_4e_{1,6},x_3e_{1,6},x_6e_{1,8},x_5e_{1,7},x_6e_{1,9},x_4x_6e_{1,5},x_4e_{1,7},\\
        & x_2e_{1,7},x_2e_{1,8},x_1e_{1,8},x_4e_{1,9},x_2x_5e_{1,4},x_2e_{1,9},x_1x_4e_{1,5}\}
    \end{aligned}
\end{equation}
Renaming each basis with second-level subscripts and applying the Lemma \ref{lem:mingens}, we have
\begin{align}
    e_{1,4}: j &= 2: (x_6,x_2x_5) = \{x_6\} \rightarrow\{x_6e_{2,17}\}\\
    e_{1,5}: j &= 2: (x_5,x_4x_6) = \{x_5\} \rightarrow\{x_5e_{2,11}\}\\
    e_{1,5}: j &= 3: ((x_5,x_4x_6):x_1x_4) = \{x_5,x_6\} \rightarrow\{x_6e_{2,19},x_5e_{2,19}\}\\
    e_{1,6}: j &= 2: (x_5:x_4) = \{x_5\} \rightarrow\{x_5e_{2,6}\}\\
    e_{1,6}: j &= 3: ((x_5,x_4):x_3) = \{x_5,x_4\} \rightarrow\{x_5e_{2,7},x_4e_{2,7}\}\\    
    e_{1,7}: j &= 2: (x_5,x_4) = \{x_5\} \rightarrow\{x_5e_{2,12}\}\\
    e_{1,7}: j &= 3: ((x_5,x_4):x_2) = \{x_5,x_4\} \rightarrow\{x_5e_{2,13},x_4e_{2,13}\}\\
    e_{1,8}: j &= 2: (x_6,x_2) = \{x_6\} \rightarrow\{x_6e_{2,14}\}\\
    e_{1,8}: j &= 3: ((x_6,x_2):x_1) = \{x_6,x_2\} \rightarrow\{x_6e_{2,15},x_2e_{2,15}\}\\
    e_{1,9}: j &= 2: (x_6,x_4) = \{x_6\} \rightarrow\{x_6e_{2,16}\}\\
    e_{1,9}: j &= 3: ((x_6,x_4):x_2) = \{x_6,x_4\} \rightarrow\{x_6e_{2,18},x_4e_{2,18}\}
\end{align}

Then, we have the third level basis $\mathcal{B}_3$ of $\Xi$:

\begin{equation}
    \begin{aligned}
        \mathcal{B}_3 = \{&x_6e_{2,17},x_5e_{2,11},x_6e_{2,19},x_5e_{2,19},x_5e_{2,6},x_5e_{2,7},x_4e_{2,7},x_5e_{2,12},x_5e_{2,13},x_4e_{2,13},x_6e_{2,14},\\
        &x_6e_{2,15},x_2e_{2,15},x_6e_{2,16},x_6e_{2,18},x_4e_{2,18}\}
    \end{aligned}
\end{equation}
After ordering it using the \verb+DegRevLex+, the third level basis is:
\begin{equation}
    \begin{aligned}
        \mathcal{B}_3 = \{&x_5e_{2,6},x_5e_{2,7},x_4e_{2,7},x_5e_{2,11},x_6e_{2,14},x_5e_{2,12},x_6e_{2,15},x_6e_{2,16},x_6e_{2,17},x_5e_{2,13},x_6e_{2,18},\\
        & x_6e_{2,19},x_4e_{2,13},x_5e_{2,19},x_2e_{2,15},x_4e_{2,18}\}
    \end{aligned}
\end{equation}

Applying the Lemma \ref{lem:mingens} once again, we find the following bases:
\begin{align}
    e_{2,19}: j &= 2: (x_6,x_5) = \{x_6\} \rightarrow\{x_6e_{3,14}\}\\
    e_{2,7}: j &= 2: (x_5,x_4) = \{x_5\} \rightarrow\{x_5e_{3,3}\}\\
    e_{2,13}: j &= 2: (x_5,x_4) = \{x_5\} \rightarrow\{x_5e_{3,13}\}\\
    e_{2,15}: j &= 2: (x_6,x_2) = \{x_6\} \rightarrow\{x_6e_{3,15}\}\\
    e_{2,18}: j &= 2: (x_6,x_4) = \{x_6\} \rightarrow\{x_6e_{3,16}\}\\
\end{align}

The fourth-level basis $\mathcal{B}_4$ of $\Xi$ is

\begin{equation}
    \begin{aligned}
        \mathcal{B}_4 = &\{x_6e_{3,14},x_5e_{3,3},x_5e_{3,13},x_6e_{3,15},x_5e_{3,16}\}
    \end{aligned}
\end{equation}
After ordering it using the \verb+DegRevLex+, the last level basis is:
\begin{equation}
    \begin{aligned}
        \mathcal{B}_4 = &\{x_5e_{3,3},x_6e_{3,14},x_5e_{3,13},x_6e_{3,15},x_5e_{3,16}\}
    \end{aligned}
\end{equation}

Now, applying the Lemma \ref{lem:mingens} could not give us any basis from $\mathcal{B}_4$, then we know that we had found the complete Schreyer frame $\Xi$.

\subsection{Computing Schreyer Resolution}

\paragraph{}

Following our plan of getting a Schreyer resolution from a Schreyer frame, we prove the following proposition.

\begin{prop}
    If $\Xi$ is a Schreyer frame for free $R$-module $N$, then there exists a Schreyer resolution $\Phi$ such that $\Xi = \initTerm (\Phi)$.
\end{prop}


\begin{proof}
    Since there $F_0$ is a free module with a ordering on it, then let $\mathcal{C}_1$ be a minimal Gröbner basis of the free module $I$, then $\initTerm(\mathcal{C}_1) = \initTerm(I)$ by the definition of the Gröbner basis. Hence, $\initTerm(\mathcal{C}_1) = \initTerm (\im \varphi_1)$, where $I = K_1 = \im \varphi_1$ referring to diagram \ref{eqn:16}. Then, $\initTerm(\mathcal{C}_1) = \initTerm (I) = \xi_1(\mathcal{E}_1) = \mathcal{B}_1$. Now, we know that for all element $m = t \cdot \epsilon \in \mathcal{B}_2$, we have $g = \varphi_1(\epsilon) \in \mathcal{C}_1$. Then using our knowledge of syzygy, we find a set of polynomials that reduces $\mathcal{B}_2$ to zero, which is the set $\mathcal{C}_2 = \varphi_2(\mathcal{E}_2)$. Since there is a term ordering on $\Xi$, the ordering on $F_1$ follows this ordering on the frame, then $\initTerm (\ker \xi_1) = \initTerm(\ker \varphi_1)$. Thus, $\mathcal{C}_2$ is a minimal Gröbner basis of $\ker (\varphi_1)$ and $\initTerm(\mathcal{C}_2) = \mathcal{B}_2$. By induction, we could construct $\mathcal{C}_i = \varphi_i(\mathcal{E}_i)$, and it is a minimal Gröbner basis of the $\ker(\varphi_{i-1})$, and $\initTerm (\varphi_i(\mathcal{E}_i)) = \initTerm(\mathcal{C}_1) = \mathcal{B_i} = \xi_i(\mathcal{E}_i)$. Thus, this gives a Schreyer resolution $\Phi$, and $\Xi = \initTerm (\Phi)$.
\end{proof}

Practically, we introduce a detailed algorithm discovered by La Scala and Stillman \cite{la_scala_strategies_1998}.

\begin{algorithm}[H]
\caption{Resolution$[\bar{{\mathcal{C}}\sb{1}}]$}\label{alg:res}
    \begin{algorithmic}
        \State Input: an irredundant Gröbner basis $\bar{{\mathcal{C}}_{1}}$ of $I$, and an ordered union $\mathcal{B}$ of bases of all levels
        \State Output: the set of Gröbner bases $\mathcal{C}_{1},\cdots,\mathcal{C}_{l}$, and the set of corresponding syzygies $\mathcal{H}_{1},\cdots,\mathcal{H}_{l}$
        \newline
        \State ${\mathcal{C}}_{i},{\mathcal{H}}_{i}:=\emptyset (1\leq i\leq l)$
        \While{$\mathcal{B}\neq\emptyset$}
            \State $m:=min\mathcal{B}$
            \State $\mathcal{B}:=\mathcal{B}\backslash\{m\}$
            \State $i:=lev(m)$
            \If{$i=1$}
                \State $g:=$ the element of $\bar{{\mathcal{C}}_{i}}$ s.t. $lm(g)=m$
                \State ${\mathcal{C}}_{1}:={\mathcal{C}}_{1}\cup\{g\}$
                \State ${\mathcal{H}}_{1}:={\mathcal{H}}_{i}\cup\{g\}$
            \State else
                \State $(f,g):=$ Reduce$[m,{\mathcal{C}}_{i-1}]$
                \State ${\mathcal{C}}_{i}:={\mathcal{C}}_{i}\cup\{g\}$
                \If{$f\neq 0$}
                    \State ${\mathcal{C}}_{i-1}:={\mathcal{C}}_{i-1}\cup\{f\}$
                    \State $\mathcal{B}:=\mathcal{B}\backslash\{lm(f)\}$
                \State else
                    \State ${\mathcal{H}}_{i}:={\mathcal{H}}_{i}\cup\{g\}$
                \EndIf
            \EndIf
        \EndWhile
        \State return ${\mathcal{C}}_{i},{\mathcal{H}}_{i}(1\leq i\leq l)$
        \end{algorithmic}
\end{algorithm}

\textcolor{BrickRed}{Talk about how to define the ordering on the resolution.}

The Algorithm \verb+Reduce+ in \verb+Resolution+ is elaborated in the Algorithm \ref{alg:red}.

\begin{algorithm}[H]
\caption{Reduce[$t\cdot \epsilon,\mathcal{C}_{i-1}]$}\label{alg:red}
    \begin{algorithmic}
    \State $f:=t\cdot k$, where $\varphi_{i-1}(\epsilon) = k$
    \State $g:=t\cdot \epsilon$
    \While{$f \neq 0$ and $lm(f) \in in\langle\mathcal{C}_{i-1}\rangle$}
        \State choose  $h \in \mathcal{C}_{i-1}$ s.t. $lm(h)\vert lm(f)$  \textbf{($h \neq f$ at the first iteration)}
        \State $f := f - \tfrac{lc(f)lpp(f)}{lc(h)lpp(h)}h$
        \State $g := g - \tfrac{lc(f)lpp(f)}{lc(h)lpp(h)}e$, where $\varphi_{i-1}(e) = h$
    \EndWhile
    \If{$f \neq 0$}
        \State $g := g - e$, where $\varphi_{i-1}(e) = f$
    \EndIf
    \State return $f,g$
    \end{algorithmic}
\end{algorithm}

\begin{example}
    Let $K$ be a field of any characteristic and let the polynomial ring $S = K[x_1,\cdots,x_6]$ with the term order of \verb+DegRevLex+. Consider the graded module $M = S/I$, for 
    \begin{equation}
        I = \langle x_2x_3,x_2x_4,x_5x_6,x_1x_2,x_4x_6,x_2x_5,x_1x_3x_5,x_1x_3x_6,x_3x_4x_5\rangle
    \end{equation} 
    Using the algorithm \verb+Resolution+ for $char(K) \neq 2$ to the module $M$, find the minimal Schreyer resolution.
\end{example}

\paragraph{}

\textbf{Solution: }Given the frame $\Xi$, we could implement the Algorithm \ref{alg:res} to construct a Schreyer resolution $\Phi$ corresponding to the frame. To start the algorithm, we should first identify the input by checking whether the first level of our frame gives a Gröbner basis. \textcolor{BrickRed}{Add more to explain the set of bases at the first level $\bar{\mathcal{C}_1}$ is a Gröbner basis}. Then we want to order the union of all bases we obtained in the frame $\Xi$.

\begin{equation}
    \begin{aligned}
        \mathcal{B} = \{&x_5x_6,x_4x_6,x_2x_5,x_1x_4,x_2x_3,x_1x_2,x_5e_{1,2},x_6e_{1,3},x_6e_{1,4},x_1x_3x_6,x_3x_4x_5,x_5e_{1,5},x_1x_3x_5, \\
                    & x_5e_{1,6},x_4e_{1,6},x_3e_{1,6},x_6e_{1,8},x_5e_{1,7},x_6e_{1,9},x_4x_6e_{1,5},x_4e_{1,7},x_2e_{1,7},x_2e_{1,8},x_1e_{1,8},x_4e_{1,9},\\
                    & x_2x_5e_{1,4},x_5e_{2,6},x_2e_{1,9},x_5e_{2,7},x_1x_4e_{1,5},x_4e_{2,7},x_5e_{2,11},x_6e_{2,14},x_5e_{2,12},x_6e_{2,15},x_6e_{2,16},\\
                    & x_6e_{2,17},x_5e_{2,13},x_6e_{2,18}, x_6e_{2,19},x_4e_{2,13},x_5e_{2,19},x_2e_{2,15},x_4e_{2,18},x_5e_{3,3},x_6e_{3,14},x_5e_{3,13},\\
                    & x_6e_{3,15},x_5e_{3,16}\}
    \end{aligned}
\end{equation}

Given $\bar{\mathcal{C}_1}$ and $\mathcal{B}$, we could apply the algorithm \verb+Resolution+. Since in Algorithm \ref{alg:res}, we just add the first-level bases in $\mathcal{B}$ to $\mathcal{C}_1$ and $\mathcal{H}_1$, then we have:

\begin{align}
        \mathcal{C}_1 &= \{x_5x_6,x_4x_6,x_2x_5,x_1x_4,x_2x_3,x_1x_2,x_1x_3x_6,x_3x_4x_5,x_1x_3x_5\}\\
        \mathcal{H}_1 &= \{x_5x_6,x_4x_6,x_2x_5,x_1x_4,x_2x_3,x_1x_2,x_1x_3x_6,x_3x_4x_5,x_1x_3x_5\}
\end{align}

Notice that since there are some bases in the second level that come before the first-level bases in the ordered list $\mathcal{B}$, we actually need to first apply the algorithm to them before applying the algorithm to the larger first-level bases. For example, when we process the basis $x_5e_{1,2}$, the $\mathcal{C}_1$ we are using is $\{x_5x_6,x_4x_6,x_2x_5,x_1x_4,x_2x_3,x_1x_2\}$. However, in this example, we do not have any bases in the second level that cannot be vanished using the smaller bases in the first level. Similarly, we also have some bases from the third level appearing before the second-level bases in the list. They are 
\begin{equation}
    x_5e_{2,6},\;x_5e_{2,7}
\end{equation}

Here, we need to be especially careful with the order in the list, since they are using different $\mathcal{C}_2$ when they are computed. For $x_5e_{2,6}$, we have

\begin{equation}
    \begin{aligned}
        \mathcal{C}_2 = \{&x_5e_{1,2}-x_4e_{1,1},x_6e_{1,3}-x_2e_{1,1},x_6e_{1,4}-x_1e_{1,2},x_5e_{1,5}-x_3e_{1,3},x_5e_{1,6}-x_1e_{1,3},\\
        & x_4e_{1,6}-x_2e_{1,4},x_3e_{1,6}-x_1e_{1,5},x_6e_{1,8}-x_3x_4e_{1,1},x_5e_{1,7}-x_1x_3e_{1,1},\\
        & x_6e_{1,9}-x_1x_3e_{1,1},x_4x_6e_{1,5}-x_2x_3e_{1,2},x_4e_{1,7}-x_1x_3e_{1,2},x_2e_{1,7}-x_1x_6e_{1,5},\\
        & x_2e_{1,8}-x_4x_5e_{1,5},x_1e_{1,8}-x_3x_5e_{1,4},x_4e_{1,9}-x_3x_4e_{1,4},x_2x_5e_{1,4}-x_4x_5e_{1,6}\}
    \end{aligned}
\end{equation}

Applying Algorithm \ref{alg:red} \verb+Reduce+ to $x_5e_{2,6}$ and the corresponding $\mathcal{C}_2$, we have $f,g$ to start with $f = -x_2x_5e_{1,4} + x_4x_5e_{1,6}$ and $g = x_5e_{2,6}$, since $\varphi(e_{2,6}) = -x_2e_{1,4}+x_4e_{1,6}$. Based on the term order on $k[x_1,x_2,x_3,x_4,x_5,x_6]$, $lm(f) = -x_2x_5e_{1,4}$. We have $lm(f) \in in\langle \mathcal{C}_2\rangle$, since $-x_2x_5e_{1,4} = (-1) \cdot x_2x_5e_{1,4}$. Therefore, in the first iteration of the while loop, we would get $h = x_2x_5e_{1,4}-x_4x_5e_{1,6}$, $f = 0$, and $g = x_5e_{2,6}+e_{2,17}$. Because $f = 0$, this pair $(f,g)$, would be input in the next step of Algorithm \ref{alg:res} \verb+Resolution+. Because $f = 0$, we could update $C_3$ and $H_3$ as follows.

\begin{equation}
    \begin{aligned}
        \mathcal{C}_3 &= \{x_5e_{2,6}+e_{2,17}\}\\
        \mathcal{H}_3 &= \{x_5e_{2,6}+e_{2,17}\}
    \end{aligned}
\end{equation}

\section{Minimalize Schreyer Resolution}

\section{Conclusion}

\paragraph{  }

\textcolor{BrickRed}{waiting for editing}

\newpage
\printbibliography

\end{document}